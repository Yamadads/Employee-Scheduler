\documentclass{beamer}

\usepackage[OT4]{polski}
\usepackage[utf8]{inputenc} 
\frenchspacing
\usetheme{Frankfurt}

\title{Automatyczne generowanie grafiku pracy}
\author{Sebastian Pawlak}
\institute[Politechnika Poznańska]
{
  Informatyka w Medycynie\\
  Wydział Informatyki Politechniki Poznańskiej
 }


\begin{document}

\begin{frame}
  \titlepage
\end{frame}

\begin{frame}{Spis prezentacji}
  \tableofcontents
\end{frame}

\section{Opis problemu}

\begin{frame}{Opis problemu}
  \begin{itemize}
  \item{Tworzenie grafiku pracy}
  \item{Wielu pracowników}
  \item{Różne zmiany}
  \item{Konieczność rozliczania czasu pracy co do minuty}
  \item{Długi okres czasu}
  \item{Liczne obostrzenia prawne}
  \item{Trudne do pogodzenia wymagania pracowników}
  \item{Mozolne ręczne układanie całości}
  \end{itemize}
\end{frame}

\section{Zapis matematyczny}

\begin{frame}{Zapis matematyczny}
  \begin{math} 
  min(max(((\sum_{d}^{ }\sum_{z}^{ }a_{dz0})-t_{0}),\\((\sum_{d}^{ }\sum_{z}^{ }a_{dz1})-t_{1}), ... , ((\sum_{d}^{ }\sum_{z}^{ }a_{dzp})-t_{p})))
  \end{math}
  \\ \\Przy ograniczeniach:\\
  \begin{math} 
  \forall_{p}(\sum_{d}^{ }\sum_{z}^{ }a_{dzp})\geqslant mindays
  \end{math}
  \\
  \begin{math} 
  \forall_{p}(\sum_{d}^{ }\sum_{z}^{ }a_{dzp})\leqslant maxdays
  \end{math}
  \\
  \begin{math} 
  \forall_{d}(\sum_{z}^{ }\sum_{p}^{ }a_{dzp})\geqslant minemployees
  \end{math}
  \\
  \begin{math} 
  \forall_{d}(\sum_{z}^{ }\sum_{p}^{ }a_{dzp})\leqslant maxemployees
  \end{math}
  \\
  \begin{math} 
  \forall_{d}\forall_{p}(\sum_{z}^{ }a_{dzp})\leqslant 1
  \end{math}
\end{frame}

\section{Wybrane technologie}

\begin{frame}{Wybrane Technologie}

\begin{block}{Java}
Uniwersalny język programowania. Umożliwia wdrożenie na wielu platformach.
\end{block}
\begin{block}{CPLEX}
Pakiet do tworzenia modeli optymalizacyjnych na potrzeby programowania matematycznego i programowania z ograniczeniami
\end{block}
\begin{block}{CSV}
Prosty format pliku, łatwy do generowania i wczytywania.
\end{block}
\begin{block}{Excel}
Znane użytkownikowi środowisko.
\end{block}

\end{frame}

\section{CPLEX}

\begin{frame}{CPLEX}
\begin{itemize}
\item{Oprogramowanie komercyjne wydawane przez IBM}
\item{Ograniczona wersja edukacyjna}
\item{Oferuje elastyczne i wydajne solvery do programowania matematycznego liniowego, całkowitoliczbowego, kwadratowego}
\item{Udostępnia biblioteki oraz API dla wielu języków programowania: Java, C\#, C++, C, Python}
\item{Umożliwia zastosowanie w środowisku Excel i Matlab oraz posiada własne dedykowane środowisko CPLEX Studio IDE}

\end{itemize}

\end{frame}


\section{Budowa systemu}

\begin{frame}{Budowa systemu}
\begin{block}{Arkusz programu excel z makrami}
Arkusz umożliwiający modyfikację danych grafiku. Posiada makra wykonujące program i wczytujące wyniki do nowego arkusza.
\end{block}
\begin{block}{Obsługa plików}
Klasy wczytujące i zapisujące dane do pliku csv.  
\end{block}
\begin{block}{Model danych}
Model przechowujący dane o grafiku i pracowników. 
\end{block}
\begin{block}{CPLEXBuilder}
Klasa wykorzystująca API CPLEX-a. Tworzy model optymalizacyjny z celem i ograniczeniami. Generowanie wyników.
\end{block}
\begin{block}{CPLEX}
Moduł CPLEX-a w postaci pliku JAR.
\end{block}
\end{frame}

\section{Materiały}

\begin{frame}{Materiały}
\begin{block}{Git}
https://github.com/Yamadads/Employee-Scheduler
\end{block}
\begin{block}{Wykorzystane artykuły}
\begin{itemize}
    \item {The staff scheduling problem: a general
model and applications,
Marta Soares Ferreira da Silva Rocha, 2013}
    \item{Metody zarządzania zasobami na przykładzie służby zdrowia
Justyna Uziałko,
Edward Radosiński, 2009}
    \item{Metoda komputerowego wspomagania wyznaczania harmonogramów pracy pojazdów trakcyjnych,
Tomasz Ambroziak, Renata Piętka, 2008}
    \item{IBM ILOG CPLEX Optimization Studio
Getting Started with Scheduling in CPLEX Studio}
\end{itemize}

\end{block}

\end{frame}

\begin{frame}{Koniec}
\begin{center}
Dziękuję za uwagę :)
\end{center}
\end{frame}

\end{document}


